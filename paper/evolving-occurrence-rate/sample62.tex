\documentclass{aastex62}

\newcommand{\Kepler}{{\it Kepler}}
\newcommand{\kepler}{\Kepler}
\newcommand{\corot}{{\it CoRoT}}
\newcommand{\Ktwo}{{\it K2}}
\newcommand{\ktwo}{\Ktwo}
\newcommand{\TESS}{{\it TESS}}
\newcommand{\tess}{{\it TESS}}
\newcommand{\LSST}{{\it LSST}}
\newcommand{\Wfirst}{{\it Wfirst}}
\newcommand{\SDSS}{{\it SDSS}}
\newcommand{\PLATO}{{\it PLATO}}
\newcommand{\Gaia}{{\it Gaia}}
\newcommand{\gaia}{{\it Gaia}}
\newcommand{\Teff}{$T_{\mathrm{eff}}$}
\newcommand{\teff}{$T_{\mathrm{eff}}$}
\newcommand{\FeH}{[Fe/H]}
\newcommand{\feh}{[Fe/H]}
\newcommand{\ie}{{\it i.e.}}
\newcommand{\eg}{{\it e.g.}}
\newcommand{\logg}{log \emph{g}}
\newcommand{\dnu}{$\Delta \nu$}
\newcommand{\numax}{$\nu_{\mathrm{max}}$}
\newcommand{\racomment}[1]{{\color{blue}#1}}


%% Tells LaTeX to search for image files in the 
%% current directory as well as in the figures/ folder.
\graphicspath{{./}{figures/}}

%% Reintroduced the \received and \accepted commands from AASTeX v5.2
\received{January 1, 2018}
\revised{January 7, 2018}
\accepted{\today}
%% Command to document which AAS Journal the manuscript was submitted to.
%% Adds "Submitted to " the arguement.
\submitjournal{ApJ}

\shorttitle{Evolving exoplanet occurrence}
\shortauthors{Angus et al.}

\begin{document}

\title{The evolving occurrence rate of exoplanets}

\correspondingauthor{Ruth Angus}
\email{ruthangus@gmail.com}

\author{Ruth Angus et al.}
\affil{Columbia University}

% ============================================================================
\begin{abstract}
% purpose
Studying the evolution of planetary systems is critical for understanding
the diverse population of exoplanets in the Milky Way.
% problem
The thousands of exoplanets detected and confirmed to date provide snapshots
of planetary systems at a range of ages and evolutionary states.
We can use these discovered exoplanets to study planet evolution, so
long as the ages of those systems are known.
Unfortunately, the ages of main sequence stars are extremely difficult to
measure precisely.
% methods
We develop new methods for inferring the ages of stars from their \kepler\
light curves and apply these methods to every GK dwarf star used for
exoplanet population analysis.
With an age for every exoplanetary system with a GK host, we are now able to
study the evolving properties of planetary systems.
In this paper we simply ask whether the overall occurrence rate of exoplanets
is constant over time, or if the rate decreases slowly, reflecting a slow
attrition rate of planetary bodies due to planet ejection via planet-planet
scattering.
% results
% conclusion
\end{abstract}

\keywords{Exoplanets: populations}

\section{Introduction}
\label{sec:introduction}
\subsection{Exoplanet Populations}
\label{sec:exopops}

Any theory of planet formation must be able to reproduce the observed
population of exoplanets in the galaxy.
In this way, exoplanet population studies provide empirical tests for planet formation theories.
Exoplanet evolutionary processes...

\subsection{Stellar ages}
\label{sec:intro_ages}

In order to characterize the population of exoplanets as a function of age,
it is necessary to infer relatively precise ages for every star in the sample.
This is a tall order as the ages of dwarf stars are extremely difficult to
infer.
High resolution spectra have been obtained for some of the bright \kepler\
stars, for example by the California Kepler Survey (CKS) \racomment{citation}.
Still, there do exist some observed properties that convey age information for
each \Kepler\ dwarf.
These properties include:
% \begin{itemize}
% \item

Data that are readily available for every Kepler dwarf that convey age
information are limited.
These

\section{Method}
\label{sec:method}
\section{The Data}
\label{sec:data}
\subsection{Inferring stellar ages}
\label{sec:stellar_ages}
\subsection{Inferring the Evolving Exoplanet Occurrence Rate}

\section{Results and Discussion}
\label{sec:discussion}

\section{Conclusion}
\label{sec:conclusion}

% acknowledgements
This research was funded by the Simons Foundation.
Some of the data presented in this paper were obtained from the Mikulski
Archive for Space Telescopes (MAST).
STScI is operated by the Association of Universities for Research in
Astronomy, Inc., under NASA contract NAS5-26555.
Support for MAST for non-HST data is provided by the NASA Office of Space
Science via grant NNX09AF08G and by other grants and contracts.
This paper includes data collected by the Kepler mission. Funding for the
Kepler mission is provided by the NASA Science Mission directorate.

\begin{thebibliography}{} 
\end{thebibliography}

\end{document}